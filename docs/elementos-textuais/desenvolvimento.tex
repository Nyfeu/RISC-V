\chapter{Desenvolvimento}
\label{cap:desenvolvimento}

\section{Metodologias de Projeto}

A complexidade inerente ao desenvolvimento de um microprocessador exige uma metodologia de projeto rigorosa e estruturada. Este trabalho seguiu uma abordagem de desenvolvimento \textbf{Bottom-Up} (de baixo para cima) combinada com \textbf{Prototipagem Evolutiva}.

\subsection{Desenvolvimento Modular e Hierárquico}

O sistema foi decomposto em módulos fundamentais de menor complexidade. O desenvolvimento iniciou-se pelos componentes base do \textit{datapath} (Multiplexadores, ALU, Banco de Registradores), progredindo para a Unidade de Controle e, finalmente, para a integração no nível de topo (\textit{Top Level}). Cada módulo foi projetado como uma entidade independente, com interfaces (\textit{ports}) claramente definidas, facilitando o isolamento de falhas.

\subsection{Estratégia de Verificação e Simulação}

Dada a dificuldade de depurar erros diretamente no hardware físico, adotou-se a abordagem de testes através de simulações (\textit{textbenches}). Nenhum componente foi integrado ao sistema principal sem antes ser validado individualmente. 

O ambiente de simulação utilizado foi o \textbf{GHDL}, um simulador \textit{open-source} de alto desempenho, integrado a um fluxo de automação via \textbf{GNU Make}. Isso garante que todo o processo de compilação, simulação e visualização de ondas (via GTKWave) seja reprodutível e automatizado.

\section{Arquitetura de Hardware}

\section{Ecossistema de Software (BARE-METAL)}

\lipsum[2-3]

% \begin{algorithm}[H]
%    \SetAlgoLined
%    \Entrada{$S,\eta, U$} 
%    \Saida{Número esperado de nodos atingidos}
%    \Inicio{
%    $\sigma(S) = 0$ \\
%     \Para{cada $u \in S$}{
%    $\sigma(S)\leftarrow \sigma(S)+\textsc{Backtrack}(u,\eta,W,U)$\\
%      }
%    }
%    \Retorna{$\sigma(S)$}
%    \label{alg1}
%    \caption{\textsc{Esperança}}
%  \end{algorithm}

\subsection{Toolchain RISC-V e Compilação}

\subsection{Automação com Makefile}

\subsection{Integração com simulação e verificações}

\section{Implementação em FPGA}

\lipsum[2-3]

\subsection{Síntese}

\subsection{Pinagem e restrições}

\subsection{Testes em hardware}

\section{Ecossistema Educacional e Ferramentas}

