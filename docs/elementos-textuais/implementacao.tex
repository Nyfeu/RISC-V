\chapter{Microarquitetura}
\label{cap:microarquitetura}

A microarquitetura define a organização lógica e física dos componentes 
internos do processador para implementar o conjunto de instruções (ISA) alvo. 
Neste capítulo, detalha-se a construção do núcleo RISC-V, focando na 
separação clássica entre o \textbf{Caminho de Dados} (\textit{Datapath}) — 
responsável pelo armazenamento e manipulação das informações — e a 
\textbf{Unidade de Controle} (\textit{Control}) — responsável por orquestrar o fluxo de dados 
através de sinais de controle.

\section{Visão Geral do Caminho de Dados}

O Caminho de Dados (\textit{Datapath}) é o 'músculo' do processador. Ele é composto por 
elementos de estado (memórias e registradores) e elementos combinacionais (ALU, somadores, 
multiplexadores). 

O fluxo principal de execução segue a seguinte sequência lógica dentro do hardware:
\begin{enumerate}
    \item \textbf{Busca (Fetch):} O Contador de Programa (PC) endereça a memória de 
    instruções.
    \item \textbf{Decodificação (Decode):} A instrução é quebrada em campos; os 
    registradores fonte são lidos e valores imediatos são estendidos.
    \item \textbf{Execução (Execute):} A ALU realiza operações aritméticas/lógicas ou calcula 
    endereços de memória.
    \item \textbf{Memória (Memory):} Dados são lidos ou escritos na Memória de Dados (se 
    aplicável).
    \item \textbf{Escrita (Write-Back):} O resultado é escrito de volta no banco de 
    registradores.
\end{enumerate}

\subsection{Unidade Lógica e Aritmética}

A Unidade Lógica e Aritmética (ALU) é um circuito puramente combinacional responsável por todas as operações matemáticas e lógicas do processador. A implementação descrita no arquivo \texttt{alu.vhd} opera com duas entradas de 32 bits e gera um resultado de 32 bits, além de \textit{flags} de estado (Zero e Negativo).

As operações suportadas foram mapeadas através de um sinal de controle de 4 bits (\texttt{ALUControl}), permitindo:

\subsection{Banco de Registradores}

\section{Implementação Single-Cycle (RV32I)}

\chapter{ORGANIZAÇÃO DE SISTEMAS E I/O}

\section{Comunicação entre CPU e Memória}
\section{Mapeamento de Memória (MMIO)}


