% Modelo de monografia criado para a Escola de Engenharia Mauá
% do Instituto Mauá de Tecnolgoia a partir da classe abntex2.
% Este documento só deverá ser alterado para incluir ou excluir
% elementos pré e pós textuais. Use o comentário do latex (%) caso
% deseje excluir algum elemento.

% --------------------------------------------------------------------------
% Definindo ABNTeX2
% --------------------------------------------------------------------------

\documentclass[
	% -- opções da classe memoir --
	12pt,				% tamanho da fonte
	openright,			% capítulos começam em pág ímpar (insere página vazia caso preciso)
	oneside,			% para impressão em frente e verso. Oposto a oneside
	a4paper,			% tamanho do papel.
	% -- opções da classe abntex2 --
	chapter=TITLE,		% títulos de capítulos convertidos em letras maiúsculas
	%section=TITLE,		% títulos de seções convertidos em letras maiúsculas
	%subsection=TITLE,	% títulos de subseções convertidos em letras maiúsculas
	%subsubsection=TITLE,% títulos de subsubseções convertidos em letras maiúsculas
	% -- opções do pacote babel --
	english,			% idioma adicional para hifenização
	brazil				% o último idioma é o principal do documento
	]{abntex2}

% --------------------------------------------------------------------------
% Pacotes básicos 
% --------------------------------------------------------------------------

\usepackage{lmodern}			% Usa a fonte Latin Modern
\usepackage{mathptmx}			% Usa a fonte Times New Roman
\usepackage[T1]{fontenc}		% Selecao de codigos de fonte.
\usepackage[utf8]{inputenc}		% Codificacao do documento (conversão automática dos acentos)
\usepackage{lastpage}			% Usado pela Ficha catalográfica
\usepackage{indentfirst}		% Indenta o primeiro parágrafo de cada seção.
\usepackage{color}				% Controle das cores
\usepackage{graphicx}			% Inclusão de gráficos
\usepackage{subcaption}				% Inclusão de gráficos lado a lado
\usepackage{microtype} 			% para melhorias de justificação
\usepackage{tabularx,ragged2e}	% Para inserir tabelas
\usepackage{multirow}			% Para mesclar células
\usepackage[dvipsnames,table,xcdraw]{xcolor}		% Permite adicionar cores nas linhas de tabelas
\usepackage{fancyvrb}			% Permite adicionar arquivos de texto
\usepackage[portuguese, ruled, linesnumbered]{algorithm2e} % Uso de algoritmos
\usepackage{amsfonts}			% Permite usar notação de conjuntos
\usepackage{amsmath}			% Permite citar equações
\usepackage{amsthm}				% Permite criar teoremas e experimentos
\usepackage[font={bf, small}, labelsep=endash, labelfont=bf]{caption}	% Faz legenda de figuras ficarem em negrito
\usepackage{cancel}				% Permite fazer expressão tendendo a zero
\usepackage{epstopdf}			% Converte eps para pdf
\usepackage[final]{pdfpages}
\usepackage{geometry}
\usepackage{longtable}
\usepackage{booktabs}
\usepackage{xltabular}

\newcolumntype{L}{>{\RaggedRight\arraybackslash}X}
\pdfstringdefDisableCommands{%
  \renewcommand{\uppercase}[1]{#1}%
}
		
% --------------------------------------------------------------------------
% Pacotes adicionais, retirar posteriormente
% --------------------------------------------------------------------------

\usepackage{lipsum}				% para geração de dummy text

% --------------------------------------------------------------------------
% Pacotes de citações
% --------------------------------------------------------------------------

\usepackage[alf, abnt-emphasize=bf]{abntex2cite}	% Citações padrão ABNT

% --------------------------------------------------------------------------
% Customizações para o layout do IMT
% --------------------------------------------------------------------------

\usepackage{modelo-maua/maua}

% Muda o título de lista de ilustrações para lista de figuras
\addto\captionsbrazil{%
  \renewcommand{\listfigurename}%
    {Lista de Ilustrações}%
	\renewcommand{\listtablename}%
    {Lista de Tabelas}%
}

% Permite utilizar figuras sem precisar colocar o caminho absoluto
\graphicspath{{imagens/}}

% Define o ambiente de experimentos
\theoremstyle{definition}
\newtheorem{experimento}{Experimento}[section]
\newcommand{\experimentoautorefname}{Experimento}

% --------------------------------------------------------------------------
% Informações de dados para CAPA e FOLHA DE ROSTO
% --------------------------------------------------------------------------

\universidade{CENTRO UNIVERSITÁRIO DO INSTITUTO MAUÁ DE TECNOLOGIA}
\faculdade{ESCOLA DE ENGENHARIA MAUÁ}
\curso{ENGENHARIA DE COMPUTAÇÃO}
\titulo{RISC-V SOC NPU}
\autor{André Solano Ferreira Rodrigues Maiolini}
\local{SÃO PAULO}
\data{2026}
\orientador{Prof. Me. Núncio Perella}
\preambulo{Aplicando conceitos de Arquitetura de Computadores para o desenvolvimento de um processador RISC-V.}

% --------------------------------------------------------------------------
% Configurações de aparência do PDF final
% --------------------------------------------------------------------------

% alterando o aspecto da cor azul
\definecolor{blue}{RGB}{41,5,195}

% informações do PDF
\makeatletter
\hypersetup{
     	%pagebackref=true,
		pdftitle={\imprimirtitulo}, 
		pdfauthor={\imprimirautor},
    	pdfsubject={\imprimirpreambulo},
	    pdfcreator={LaTeX with abnTeX2},
		pdfkeywords={\imprimirpalavraschave}, 
		colorlinks=true,       		% false: boxed links; true: colored links
    	linkcolor=black,          	% color of internal links
    	citecolor=black,        		% color of links to bibliography
    	filecolor=magenta,      		% color of file links
		urlcolor=blue,
		bookmarksdepth=4,
        breaklinks=true
}
\makeatother

% --------------------------------------------------------------------------
% Espaçamentos entre linhas e parágrafos 
% --------------------------------------------------------------------------

% O tamanho do parágrafo é dado por:
\setlength{\parindent}{1.3cm}

% Controle do espaçamento entre um parágrafo e outro:
\setlength{\parskip}{0.2cm}  % tente também \onelineskip

% --------------------------------------------------------------------------
% Compila o indice
% --------------------------------------------------------------------------

\makeindex

% --------------------------------------------------------------------------
% Início do documento
% --------------------------------------------------------------------------

\begin{document}

% Seleciona o idioma do documento (conforme pacotes do babel)
%\selectlanguage{english}
\selectlanguage{brazil}

% Retira espaço extra obsoleto entre as frases.
\frenchspacing 

% ----------------------------------------------------------
% ELEMENTOS PRÉ-TEXTUAIS
% ----------------------------------------------------------

% --------------------------------------------------------------------------
% Capa
% --------------------------------------------------------------------------

\imprimircapa

% --------------------------------------------------------------------------
% Folha de rosto
% --------------------------------------------------------------------------

\imprimirfolhaderosto*
\newpage

% --------------------------------------------------------------------------
% Dedicatória
% --------------------------------------------------------------------------

\begin{dedicatoria}
   \vspace*{\fill}
   \centering
   \noindent
   \textit{Este trabalho é dedicado àqueles cuja curiosidade os\\empurra sempre um pouco além do necessário.} \vspace*{\fill}
\end{dedicatoria}

% --------------------------------------------------------------------------
% Agradecimentos
% --------------------------------------------------------------------------

% \begin{agradecimentos}
%
% Reescrever!
%
% \end{agradecimentos}

% --------------------------------------------------------------------------
% Epígrafe
% --------------------------------------------------------------------------

\begin{epigrafe}
    \vspace*{\fill}
	\begin{flushright}
        \begin{figure}[!htb]
        	\label{fig:dragon}
            \raggedleft
        	\includegraphics[width=.5\textwidth]{imagens/dragon.jpg}
        \end{figure}
		\textit{``Fairy tales are more than true: not because they\\
                  tell us that dragons exist, but because they tell us\\
                  that dragons can be beaten.''\\
		(Neil Gaiman)}
	\end{flushright}
\end{epigrafe}

% --------------------------------------------------------------------------
% Inserir listas - figuras, tabelas, algoritmos, quadros, sumário etc.
% --------------------------------------------------------------------------

% --------------------------------------------------------------------------
% Inserir lista de ilustrações
% --------------------------------------------------------------------------

\pdfbookmark[0]{\listfigurename}{lof}
\listoffigures*
\cleardoublepage

% --------------------------------------------------------------------------
% Inserir lista de quadros
% --------------------------------------------------------------------------
\pdfbookmark[0]{\listofquadrosname}{loq}
\listofquadros*
\cleardoublepage

% --------------------------------------------------------------------------
% Inserir lista de tabelas
% --------------------------------------------------------------------------
\pdfbookmark[0]{\listtablename}{lot}
\listoftables*
\cleardoublepage

% --------------------------------------------------------------------------
% Inserir lista de algoritmos
% --------------------------------------------------------------------------
\pdfbookmark[0]{\listalgorithmcfname}{loa}
\imprimirlistadealgoritmos
\cleardoublepage

% --------------------------------------------------------------------------
% Inserir lista de abreviaturas e siglas
% --------------------------------------------------------------------------
\begin{siglas}
	\item[ABNT] Associação Brasileira de Normas Técnicas
  \item[abnTeX] ABsurdas Normas para TeX
\end{siglas}

% --------------------------------------------------------------------------
% Inserir lista de símbolos
% --------------------------------------------------------------------------
\begin{simbolos}
	\item[$ \Gamma $] Letra grega Gama
  \item[$ \Lambda $] Lambda
  \item[$ \zeta $] Letra grega minúscula zeta
  \item[$ \in $] Pertence
\end{simbolos}

% --------------------------------------------------------------------------
% Inserir o sumario
% --------------------------------------------------------------------------
\pdfbookmark[0]{\contentsname}{toc}
\tableofcontents*
\cleardoublepage

% ----------------------------------------------------------
% ELEMENTOS TEXTUAIS
% ----------------------------------------------------------

\textual

% ----------------------------------------------------------
% Introdução
% ----------------------------------------------------------

\chapter{Introdução}
\label{cap:introducao}

A computação moderna é construída sobre camadas de abstração. Do clique de um mouse à 
execução de uma inteligência artificial, existem bilhões de transistores operando em uma 
sinfonia lógica precisa. No entanto, para a maioria dos profissionais da área, essa base 
opera como uma "caixa-preta". O hardware tornou-se algo que se usa, mas não necessariamente 
se compreende.

Este trabalho propõe romper com essa visão utilitária. O projeto aqui documentado não é apenas 
a construção de um microprocessador, mas, uma investigação prática sobre os fundamentos que 
sustentam toda a era digital. Ao projetar, implementar e validar um sistema computacional 
completo, buscamos resgatar o domínio sobre a camada de hardware subjacente.

A arquitetura escolhida para guiar esta jornada é o RISC-V, um conjunto de instruções (ISA) 
aberto que, por sua simplicidade e modularidade, serve como o veículo perfeito para o estudo de 
arquitetura e organização de computadores.

\section{Motivação}

Por que, na era da computação em nuvem e linguagens de altíssimo nível, um engenheiro deve 
despender tempo entendendo barramentos, datapaths e sinais de controle? A despeito do avanço das 
abstrações oferecidas por linguagens modernas, frameworks e plataformas em nuvem, os sistemas 
computacionais continuam fundamentados em princípios microarquiteturais que determinam seu 
desempenho, seu comportamento e suas limitações. Esses conhecimentos permanecem essenciais porque 
tais componentes constituem o nível de base sobre o qual todas as abstrações são 
construídas. Sem essa compreensão, o engenheiro passa a operar máquinas sem entender suas 
características - tais como: latência, gargalos, paralelismo, hierarquia de memória e restrições — 
que afetam diretamente a eficiência e a confiabilidade de qualquer solução de software ou hardware 
desenvolvida.

Do ponto de vista formativo, a familiaridade com os elementos internos do processador também 
capacita o engenheiro a compreender o funcionamento da cadeia completa de execução — da 
especificação de uma instrução ao seu efeito no hardware físico. Essa visão abrangente é 
indispensável em áreas como sistemas embarcados, computação de alto desempenho, aceleradores 
especializados, dispositivos IoT, aplicações críticas e de tempo real, nas quais a interação 
direta com o hardware é frequente e decisiva.

Em suma, mesmo em um cenário em que as abstrações se tornam cada vez mais sofisticadas, a 
compreensão do nível fundamental da computação permite que o engenheiro avalie limitações, 
explore otimizações, antecipe comportamentos do sistema e desenvolva soluções mais robustas e 
eficientes. Assim, o estudo desses conceitos continua sendo um elemento central na formação e 
na prática da engenharia moderna.

\section{Objetivos do Projeto}

O objetivo central deste trabalho é suprir o abismo entre estudar diagramas de livros-texto e 
fazer um sinal elétrico acender um pixel na tela. Para tal, projetando um System-on-Chip (SoC) 
baseado na arquitetura RISC-V, validando os conceitos teóricos de organização de computadores 
através de uma aplicação prática. 

Portanto, visando prover uma experiência prática acerca dos tópicos estudados, os objetivos 
específicos desse projeto são:

\begin{enumerate}
    \item \textbf{Microarquitetura}: Implementar um núcleo de processamento funcional, detalhando o 
    fluxo de dados e a lógica de controle.
    \item \textbf{Organização de Sistemas}: Desenvolver uma infraestrutura de comunicação 
    (Barramento) que demonstre na prática o conceito de Mapeamento de Memória (MMIO), permitindo 
    que a CPU interaja com o mundo externo.
    \item \textbf{Interface Hardware/Software}: Construir o ecossistema necessário para executar 
    código C em um ambiente sem sistema operacional (\textit{bare-metal}), desmistificando o funcionamento 
    de bibliotecas padrão.
\end{enumerate}

\section{Organização do Documento}

Este documento está organizado de forma a acompanhar evolução do projeto - abordando a concepção 
da arquitetura e desenvolvimento da microarquitetura, organização do sistema, até a execução de 
aplicações e simulações:

\begin{enumerate}
    \item O \textbf{Capítulo 2} fundamenta as escolhas de arquitetura (RISC-V) e ferramentas.
    \item O \textbf{Capítulo 3} descreve o coração do sistema: o processador e sua microarquitetura.
    \item O \textbf{Capítulo 4} expande o horizonte para o sistema completo, detalhando o Barramento e a organização da memória.
    \item O \textbf{Capítulo 5} aborda a camada invisível de software que torna o hardware utilizável.
    \item O \textbf{Capítulo 6} apresenta a implementação em um hardware real, através de uma placa FPGA, consolidando todos os conceitos em uma aplicação real.
\end{enumerate}


% Capitulo com exemplos de comandos inseridos de arquivo externo 
\include{abntex2-modelo-include-comandos}

% ----------------------------------------------------------
% Revisão da Literatura
% ----------------------------------------------------------

\chapter{Fundamentação}
\label{cap:fundamentacao}

% ---
\section{Arquitetura RISC-V}
% ---

O RISC-V é uma ISA (Instruction Set Architecture) baseada no conceito RISC (Reduced Instruction 
Set Computer). Diferente de arquiteturas proprietárias (como x86 e ARM), a RISC-V é aberta e 
livre de royalties, permitindo que universidades, empresas e desenvolvedores a utilizem e 
modifiquem. 

\begin{figure}[!htb]
	\caption{Arquitetura geral de um núcleo RISC-V, destacando o \textit{pipeline} de execução, o banco de registradores e as interfaces de memória.}
	\label{fig:risc-v-architecture}
	\centering
	\includegraphics[width=\textwidth]{risc-v-architecture.jpg} \\
	\begin{small}\textbf{Fonte: O Autor (2017)}\end{small}
\end{figure}

As principais características da arquitetura RISC-V incluem:
\begin{enumerate}
	\item Modularidade: núcleo mínimo (base ISA) e extensões opcionais.
	\item Simplicidade: instruções de tamanho fixo (geralmente 32 bits).
	\item Portabilidade: projetada para uso desde microcontroladores até supercomputadores.
	\item Ecossistema aberto: suporte crescente em compiladores, simuladores e hardware.
\end{enumerate}

Apesar do domínio da ARM no mercado de embarcados e móveis, a flexibilidade, o custo zero de 
licenciamento e a capacidade de customização do RISC-V o posicionam como um competidor 
formidável e uma alternativa estratégica para o futuro da computação.

Assim como processadores de arquitetura ARM (\textit{Advanced RISC Machine}), os RISC-V recebem 
esse nome por serem baseados nos princípios RISC - computação com conjunto de instruções 
reduzido (\textit{Reduced Instruction Set Computing}). 

Diferentemente da arquitetura CISC (\textit{Complex Instruction Set Computing}), que se baseia 
em um vasto conjunto de instruções complexas e especializadas, o RISC-V adota a filosofia do 
"menos é mais". Ao simplificar o conjunto de instruções, a arquitetura permite um hardware mais 
eficiente e previsível.

\subsection{Conjunto de Instruções RV32I}

O conjunto de instruções \textbf{RV32I} é a especificação base da arquitetura RISC-V para 
processadores de 32 bits. O nome indica que é um processador RISC-V de 32 bits de largura da 
palavra de processamento (\textit{word}) e utiliza a extensão de números inteiros 
(\textit{Integer}) - que é a base para todas as outras. A Tabela \ref{tab:rv32i_instr} lista as instruções base suportadas pelo núcleo RV32I.

\begin{xltabular}{\textwidth}{@{} l l X @{}}
\caption{Instruções do Conjunto RV32I} \label{tab:rv32i_instr} \\
\toprule
\textbf{Categoria} & \textbf{Mnemônico} & \textbf{Descrição} \\
\midrule
\endfirsthead
\toprule
\textbf{Categoria} & \textbf{Mnemônico} & \textbf{Descrição} \\
\midrule
\endhead
\bottomrule
\endfoot

% Aritmética e Lógica (R-Type)
\textbf{R-Type} & ADD & Soma aritmética entre registadores. \\
 & SUB & Subtração aritmética entre registadores. \\
 & SLL & Deslocamento lógico à esquerda (\textit{Shift Left Logical}). \\
 & SLT & Definir se menor que (\textit{Set Less Than}, com sinal). \\
 & SLTU & Definir se menor que (\textit{Set Less Than Unsigned}, sem sinal). \\
 & XOR & Ou exclusivo bit a bit. \\
 & SRL & Deslocamento lógico à direita (\textit{Shift Right Logical}). \\
 & SRA & Deslocamento aritmético à direita (\textit{Shift Right Arithmetic}). \\
 & OR & Ou lógico bit a bit. \\
 & AND & E lógico bit a bit. \\
\midrule

% Aritmética e Lógica Imediata (I-Type)
\textbf{I-Type} & ADDI & Soma com valor imediato. \\
 & SLTI & Definir se menor que imediato (com sinal). \\
 & SLTIU & Definir se menor que imediato (sem sinal). \\
 & XORI & Ou exclusivo imediato. \\
 & ORI & Ou lógico imediato. \\
 & ANDI & E lógico imediato. \\
 & SLLI & Deslocamento lógico à esquerda imediato. \\
 & SRLI & Deslocamento lógico à direita imediato. \\
 & SRAI & Deslocamento aritmético à direita imediato. \\
\midrule

% Upper Immediates (U-Type)
\textbf{U-Type} & LUI & Carregar Imediato Superior (\textit{Load Upper Immediate}). Preenche os 20 bits mais significativos. \\
 & AUIPC & Adicionar Imediato Superior ao PC (\textit{Add Upper Immediate to PC}). \\
\midrule

% Saltos (J-Type / I-Type)
\textbf{Jumps} & JAL & Salto e Link (\textit{Jump and Link}). Salto incondicional relativo ao PC. \\
 & JALR & Salto e Link Registrador (\textit{Jump and Link Register}). Salto incondicional via registrador. \\
\midrule

% Desvios Condicionais (B-Type)
\textbf{Branches} & BEQ & Desvio se Igual (\textit{Branch if Equal}). \\
 & BNE & Desvio se Diferente (\textit{Branch if Not Equal}). \\
 & BLT & Desvio se Menor (\textit{Branch if Less Than}, com sinal). \\
 & BGE & Desvio se Maior ou Igual (\textit{Branch if Greater/Equal}, com sinal). \\
 & BLTU & Desvio se Menor (\textit{Branch if Less Than}, sem sinal). \\
 & BGEU & Desvio se Maior ou Igual (\textit{Branch if Greater/Equal}, sem sinal). \\
\midrule

% Acesso à Memória (Loads)
\textbf{Loads} & LB & Carregar Byte (estende sinal). \\
 & LH & Carregar Meia-Palavra (\textit{half-word}, estende sinal). \\
 & LW & Carregar Palavra (\textit{word}, 32 bits). \\
 & LBU & Carregar Byte (sem sinal, preenche com zeros). \\
 & LHU & Carregar Meia-Palavra (sem sinal, preenche com zeros). \\
\midrule

% Acesso à Memória (Stores)
\textbf{Stores} & SB & Armazenar Byte. \\
 & SH & Armazenar Meia-Palavra. \\
 & SW & Armazenar Palavra. \\
\midrule

% Sistema e Sincronização
\textbf{System} & FENCE & Sincronização de memória e I/O. \\
 & ECALL & Chamada de sistema (\textit{Environment Call}). \\
 & EBREAK & Paragem para depuração (\textit{Environment Break}). \\

\end{xltabular}

É a ISA mais simples e fundamental, servindo como ponto de partida obrigatório. Sozinha, ela 
é suficiente para a criação de um processador de propósito geral e para a execução de um 
código compilado em C completo - usando a toolchain da RISC-V.

Um processador que se diz compatível com \textbf{RV32IMC}, por exemplo, implementa a base de 
inteiros (I), as operações de multiplicação/divisão (M) e instruções comprimidas de 16 bits (C).

\subsection{Extensões relevantes}

A arquitetura RISC-V distingue-se pela sua modularidade. O conjunto de instruções base (como o 
RV32I implementado neste trabalho) é a única parte obrigatória; sobre ela, projetistas podem 
acoplar módulos opcionais para otimizar o desempenho em domínios específicos.

As extensões padronizadas são identificadas por letras sufixas à nomenclatura do processador. 
As principais são:

\begin{itemize}

    \item \textbf{M (\textit{Integer Multiplication and Division})}: Adiciona hardware dedicado para 
	multiplicação e divisão de inteiros. Sem esta extensão, estas operações devem ser realizadas 
	via software (rotinas de emulação), o que é significativamente mais lento.
    
    \item \textbf{A (\textit{Atomic})}: Fornece instruções de leitura-modificação-escrita atômicas. 
	Estas são essenciais para a sincronização de memória e gerenciamento de exclusão mútua em 
	sistemas operacionais e arquiteturas \textit{multicore}.
    
    \item \textbf{F/D (\textit{Single/Double Precision Floating-Point})}: Adiciona um banco de 
	registradores separado e unidades de cálculo para operações com ponto flutuante, seguindo 
	o padrão IEEE 754 (32 bits para F e 64 bits para D).
    
    \item \textbf{C (\textit{Compressed})}: Introduz um conjunto de instruções de 16 bits que mapeiam 
	para as operações mais comuns de 32 bits. Isso aumenta a densidade de código, reduzindo o 
	tamanho dos binários e o uso de largura de banda de memória.
    
    \item \textbf{V (\textit{Vector})}: Habilita operações vetoriais (SIMD - \textit{Single Instruction, 
	Multiple Data}), permitindo o processamento paralelo de grandes volumes de dados, comum em 
	aplicações de IA e processamento de sinal.
    
    \item \textbf{B (\textit{Bit Manipulation})}: Oferece instruções eficientes para manipulação de 
	bits individuais (rotações, contagens, extrações), úteis em criptografia e drivers de 
	dispositivos.

\end{itemize}

É comum na literatura encontrar a designação \textbf{RV32G} ou \textbf{RV64G}, onde 'G' 
(\textit{General}) representa um agrupamento das extensões mais utilizadas em sistemas de 
propósito geral.

Além do padrão, a ISA reserva espaço de opcodes para \textbf{extensões customizadas (X)}. Esta 
característica permite a criação de aceleradores de hardware proprietários para tarefas 
especializadas (domain-specific architectures) sem quebrar a compatibilidade com o ecossistema 
de software base.

\section{VHDL e Projeto de Hardware}

A descrição de hardware deste projeto foi realizada utilizando a linguagem \textbf{VHDL} 
(\textit{VHSIC Hardware Description Language}). Diferente de linguagens de programação de 
software, que descrevem sequências de instruções, o VHDL é uma linguagem de descrição de 
hardware utilizada para modelar o comportamento concorrente e a estrutura de circuitos digitais.

Para este trabalho, adotou-se especificamente o padrão \textbf{VHDL-2008} (IEEE 1076-2008). 
Esta revisão da linguagem introduz construções modernas que aumentam a legibilidade e a 
capacidade de síntese do código, permitindo uma descrição mais concisa de lógicas complexas 
sem sacrificar o rigor tipológico característico da linguagem.

\section{Metodologias de Projeto}

A complexidade inerente ao desenvolvimento de um microprocessador exige uma metodologia de 
projeto rigorosa e estruturada. Este trabalho seguiu uma abordagem de desenvolvimento 
\textbf{Bottom-Up} (de baixo para cima) combinada com \textbf{Prototipagem Evolutiva}.

\subsection{Desenvolvimento Modular e Hierárquico}
O sistema foi decomposto em módulos fundamentais de menor complexidade. O desenvolvimento 
iniciou-se pelos componentes base do \textit{datapath} (Multiplexadores, ALU, Banco de 
Registradores), progredindo para a Unidade de Controle e, finalmente, para a integração no 
nível de topo (\textit{Top Level}). Cada módulo foi projetado como uma entidade independente, 
com interfaces (\textit{ports}) claramente definidas, facilitando o isolamento de falhas.

\subsection{Estratégia de Verificação e Simulação}
Dada a dificuldade de depurar erros diretamente no hardware físico, adotou-se a abordagem de 
testes através de simulações (\textit{textbenches}). Nenhum componente foi integrado ao sistema 
principal sem antes ser validado individualmente. 

O ambiente de simulação utilizado foi o \textbf{GHDL}, um simulador \textit{open-source} de alto  
desempenho, integrado a um fluxo de automação via \textbf{GNU Make}. Isso garante que todo o 
processo de compilação, simulação e visualização de ondas (via GTKWave) seja reprodutível e 
automatizado.

% ----------------------------------------------------------
% Implementação
% ----------------------------------------------------------

\chapter{Microarquitetura}
\label{cap:microarquitetura}

A microarquitetura define a organização lógica e física dos componentes 
internos do processador para implementar o conjunto de instruções (ISA) alvo. 
Neste capítulo, detalha-se a construção do núcleo RISC-V, focando na 
separação clássica entre o \textbf{Caminho de Dados} (\textit{Datapath}) — 
responsável pelo armazenamento e manipulação das informações — e a 
\textbf{Unidade de Controle} (\textit{Control}) — responsável por orquestrar o fluxo de dados 
através de sinais de controle.

\section{Visão Geral do Caminho de Dados}

O Caminho de Dados (\textit{Datapath}) é o 'músculo' do processador. Ele é composto por 
elementos de estado (memórias e registradores) e elementos combinacionais (ALU, somadores, 
multiplexadores). 

O fluxo principal de execução segue a seguinte sequência lógica dentro do hardware:
\begin{enumerate}
    \item \textbf{Busca (Fetch):} O Contador de Programa (PC) endereça a memória de 
    instruções.
    \item \textbf{Decodificação (Decode):} A instrução é quebrada em campos; os 
    registradores fonte são lidos e valores imediatos são estendidos.
    \item \textbf{Execução (Execute):} A ALU realiza operações aritméticas/lógicas ou calcula 
    endereços de memória.
    \item \textbf{Memória (Memory):} Dados são lidos ou escritos na Memória de Dados (se 
    aplicável).
    \item \textbf{Escrita (Write-Back):} O resultado é escrito de volta no banco de 
    registradores.
\end{enumerate}

\subsection{Unidade Lógica e Aritmética}

A Unidade Lógica e Aritmética (ALU) é um circuito puramente combinacional responsável por todas as operações matemáticas e lógicas do processador. A implementação descrita no arquivo \texttt{alu.vhd} opera com duas entradas de 32 bits e gera um resultado de 32 bits, além de \textit{flags} de estado (Zero e Negativo).

As operações suportadas foram mapeadas através de um sinal de controle de 4 bits (\texttt{ALUControl}), permitindo:

\subsection{Banco de Registradores}

\section{Implementação Single-Cycle (RV32I)}

\chapter{ORGANIZAÇÃO DE SISTEMAS E I/O}

\section{Comunicação entre CPU e Memória}
\section{Mapeamento de Memória (MMIO)}



\chapter{Ecossistema de Software (BARE-METAL)}
\label{cap:ecossistema-sw}

\lipsum[2-3]

% \begin{algorithm}[H]
%    \SetAlgoLined
%    \Entrada{$S,\eta, U$} 
%    \Saida{Número esperado de nodos atingidos}
%    \Inicio{
%    $\sigma(S) = 0$ \\
%     \Para{cada $u \in S$}{
%    $\sigma(S)\leftarrow \sigma(S)+\textsc{Backtrack}(u,\eta,W,U)$\\
%      }
%    }
%    \Retorna{$\sigma(S)$}
%    \label{alg1}
%    \caption{\textsc{Esperança}}
%  \end{algorithm}

\section{Toolchain RISC-V e Compilação}

\section{Automação com Makefile}

\section{Integração com simulação e verificações}


\chapter{Implementação em FPGA}
\label{cap:fpga}

\lipsum[2-3]

\section{Síntese}

\section{Pinagem e restrições}

\section{Testes em hardware}



% ----------------------------------------------------------
% Considerações Finais
% ----------------------------------------------------------

\chapter{Conclusão}
\label{conclusao}

\lipsum[31-33]

% ----------------------------------------------------------
% ELEMENTOS PÓS-TEXTUAIS
% ----------------------------------------------------------


\postextual

% ----------------------------------------------------------
% Referências bibliográficas
% ----------------------------------------------------------

\bibliography{bibliografia}

% ----------------------------------------------------------
% Apêndices
% ----------------------------------------------------------

% \begin{apendicesenv}
	
	% Imprime uma página indicando o início dos apêndices
%	\partapendices
%   % -------------------------------------------
% Nome do Apêndice: TEMPLATE
% -------------------------------------------

\chapter{Quisque libero justo}

\lipsum[50]
%   % -------------------------------------------
% Nome do Apêndice: TEMPLATE
% -------------------------------------------

\chapter{Nullam elementum urna}

\lipsum[55-57]
	
%\end{apendicesenv}


% ----------------------------------------------------------
% Anexos
% ----------------------------------------------------------

% \begin{anexosenv}
	
	% Imprime uma página indicando o início dos anexos
%	\partanexos
%	% -------------------------------------------
% Nome do Anexo: TEMPLATE
% -------------------------------------------

\chapter{Morbi ultrices rutrum lorem}

\lipsum[30]


%   \input{elementos-pos-textuais/anexos/anexo_b.tex}
	
% \end{anexosenv}

\end{document}